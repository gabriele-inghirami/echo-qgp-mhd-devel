\documentclass[11pt,a4paper]{report}
\usepackage[english]{babel}

% math stuff, symbols --------------------------------------
\usepackage{mathrsfs} % fancy math letters
\usepackage{amsmath}
\usepackage{amsfonts}
\usepackage{amssymb}
\usepackage{amsthm}
\usepackage{mathdesign}
\usepackage{slashed}
\usepackage{mathtools}% \usepackage{empheq} is included in mathtools
%------------------------------------------------------------
\usepackage{graphicx}
\usepackage[usenames,dvipsnames,svgnames,table]{xcolor} % more shades than color
\usepackage{fancyhdr}

% caption --------------------------------------------------
\usepackage[font={sl,footnotesize},format=hang,labelfont={bf,sl,small},width=0.8\textwidth]{caption}
\usepackage{subcaption}


% itemize/description/enumerate ----------------------------
\usepackage{enumitem} %pro­vides user con­trol over the lay­out of the three ba­sic list en­vi­ron­ments

% Miscellanea ----------------------------------------------
\usepackage{cite}
\usepackage{calc}
%% LaTeX can manipulate numbers.The calc package provides the common infix notation.
\usepackage{lscape} 
% Set a page in landscape with	\begin{landscape}
\usepackage{centernot}
% prints the sym­bol \not on the fol­low­ing ar­gu­ment. 
% Un­like the de­fault \not com­mand, the sym­bol is hor­i­zon­tally cen­tered. 
\usepackage{mhchem} % Chemistry formulas, Isotopes, etc. 
% see usage here: https://docs.moodle.org/27/en/Chemistry_notation_using_mhchem
\usepackage{listings}%To insert programming code within the document. 
\lstset{breaklines=true}
% Many languages are supported and the output can be customized.

\usepackage[     %pdftex%, 
                 %plainpages = false, %pdfpagelabels, 
                 %pdfpagelayout = useoutlines,
                 bookmarks,
                 bookmarksopen = true,
                 bookmarksnumbered = true,
                 breaklinks = true,
                 linktocpage,
                 %pagebackref,
                 colorlinks = true,  % was true
                 linkcolor = blue,
                 urlcolor  = black,
                 citecolor = blue,
                 anchorcolor = black,
                 hyperindex = true,
                 %hyperfigures
                 ]{hyperref} % manage links within the document or to any URL when you compile in PDF
                 


\usepackage{lineno} %% line numbers 
\usepackage{siunitx} %% Metric international system units


\definecolor{mygreen}{rgb}{0,0.6,0}
\definecolor{mygray}{rgb}{0.5,0.5,0.5}
\definecolor{mymauve}{rgb}{0.58,0,0.82}

\def\name{Gabriele Inghirami and Valentina Rolando}

\lstset{ %
  backgroundcolor=\color{white},   % choose the background color; you must add \usepackage{color} or \usepackage{xcolor}
  basicstyle=\footnotesize,        % the size of the fonts that are used for the code
  breakatwhitespace=false,         % sets if automatic breaks should only happen at whitespace
  breaklines=true,                 % sets automatic line breaking
  captionpos=b,                    % sets the caption-position to bottom
  commentstyle=\color{mygreen},    % comment style
%   deletekeywords={number},            % if you want to delete keywords from the given language
  escapeinside={\%*}{*)},          % if you want to add LaTeX within your code
  extendedchars=true,              % lets you use non-ASCII characters; for 8-bits encodings only, does not work with UTF-8
  frame=single,                    % adds a frame around the code
  keepspaces=true,                 % keeps spaces in text, useful for keeping indentation of code (possibly needs columns=flexible)
  keywordstyle=\color{blue},       % keyword style
  language=fortran,                 % the language of the code
%   morekeywords={*,...},            % if you want to add more keywords to the set
  numbers=left,                    % where to put the line-numbers; possible values are (none, left, right)
  numbersep=5pt,                   % how far the line-numbers are from the code
  numberstyle=\tiny\color{mygray}, % the style that is used for the line-numbers
  rulecolor=\color{black},         % if not set, the frame-color may be changed on line-breaks within not-black text (e.g. comments (green here))
  showspaces=false,                % show spaces everywhere adding particular underscores; it overrides 'showstringspaces'
  showstringspaces=false,          % underline spaces within strings only
  showtabs=false,                  % show tabs within strings adding particular underscores
  stepnumber=2,                    % the step between two line-numbers. If it's 1, each line will be numbered
  stringstyle=\color{mymauve},     % string literal style
  tabsize=2,                       % sets default tabsize to 2 spaces
  title=\lstname                   % show the filename of files included with \lstinputlisting; also try caption instead of title
}



\lstdefinestyle{customtxt}{
  belowcaptionskip=1\baselineskip,
  breaklines=true,
  frame=L,
  xleftmargin=\parindent,
%   language=none,
  showstringspaces=false,
  basicstyle=\footnotesize\ttfamily,
%   keywordstyle=\bfseries\color{green!40!black},
%   commentstyle=\itshape\color{purple!40!black},
%   identifierstyle=\color{blue},
%   stringstyle=\color{orange},
  keywordstyle=\color{black},
  commentstyle=\color{black},
  identifierstyle=\color{black},
  stringstyle=\color{black},
}
\newcommand{\Cred}{\color{red}}
\newcommand{\Cblue}{\color{blue}}
\newcommand*\mygraybox[1]{\colorbox{lightgray}{#1}}
\newcommand{\fo}{Freeze-Out }
\newcommand{\de}{\mathrm{d}}
\newcommand{\mt}{m_\mathrm{T}}
\newcommand{\pt}{p_\mathrm{T}}
\newcommand{\rap}{\mathrm{y}}
\newcommand{\dVtau}{\de V^{\perp \tau}}
\newcommand{\dVx}{\de V^{\perp x}}
\newcommand{\dVy}{\de V^{\perp y}}
\newcommand{\dVeta}{\de V^{\perp \eta}}
\newcommand{\chr}[3]{\Gamma^{#1}_{#2#3}}
\hyphenation{ba-rio-nic}
\hyphenation{dif-fe-ren-tial}
\newenvironment{sistema}%
{\left\lbrace\begin{array}{@{}l@{}}}%
{\end{array}\right.}
% The following metadata will show up in the PDF properties
\newcommand{\citneed}{{\color{red} citation needed }}
\newcommand{\refneed}{{\color{red} reference needed }}
\newcommand{\integer}{{\tt \color{PineGreen} integer}}
\newcommand{\real}{{\tt \color{PineGreen} real}}
\newcommand{\chara}{{\tt \color{PineGreen} character}}
% Other definitions:
\def\x{{\boldsymbol x}}
